% Formato de portafolio
% Documento adaptado de https://github.com/PlasmaPhysicsUdeC/FondecyTeX por Roberto Navarro <roberto.navarro@udec.cl>

% Este documento es modular, es decir, esta separado en varios
% archivos para facilitar su uso. Ver uso del paquete `subfiles`:
% https://www.overleaf.com/learn/latex/Multi-file_LaTeX_projects

\documentclass[twoside]{report}
\usepackage[table]{xcolor}
\usepackage[paper=letterpaper,left=3cm,right=3cm,top=3cm,bottom=2cm,includefoot]{geometry}
\usepackage{fancyhdr}
\usepackage{tabularx}
\usepackage{multirow}
\usepackage[colorlinks]{hyperref}
\hypersetup{
    citecolor=cyan!80!black,
    urlcolor=cyan
}
\usepackage{hhline}
\usepackage{amsmath}
\usepackage{enumitem}
\setenumerate{itemsep=-3pt,topsep=3pt}
\setdescription{itemsep=-3pt,topsep=3pt,leftmargin=!,labelwidth=4.5cm}
\usepackage{pgfgantt}
\usepackage{graphicx}
\graphicspath{{./img/} {./tex/img/} {../img/} }

\usepackage[spanish]{babel}
\usepackage[utf8]{inputenc}
\usepackage[T1]{fontenc}

% % bibliografia: descomente estas dos lineas para usar estilo numerico, e.g. [1].
% \usepackage[square,numbers,sort&compress]{natbib}
% \bibliographystyle{apsrev4-1}

% bibliografia: descomente estas dos lineas para usar estilo autor-año, e.g. (Navarro, 2022).
\usepackage[authoryear]{natbib}
\bibliographystyle{aipauth4-1}


% Para incluir codigos python.
% Necesita opcion -shell-escape para compilar
\usepackage[cachedir=/tmp/minted-caches]{minted}
\newminted[pythoncode]{python}{linenos,breaklines,mathescape,texcomments,xleftmargin=\parindent,numbersep=5pt,bgcolor=lightgray!40,fontsize=\small}

% subfiles permite que el documento sea modular
\usepackage{subfiles}

%%% Comente/Descomente las siguientes lineas para cambiar la fuente del texto
% \usepackage{DejaVuSans}
% \renewcommand*\familydefault{\sfdefault}
% \usepackage{sansmath}
% \sansmath

% \numberwithin{equation}{chapter}

\pagestyle{fancy}
\renewcommand{\footrulewidth}{0.4pt}
\renewcommand{\headrulewidth}{0.4pt}
\fancyfoot{}
\fancyfoot[RE,RO]{\thepage}
\fancyfoot[LO,LE]{\textcolor[RGB]{127,127,127}{Portafolio - Física Computacional II (2024)}}

\fancyhead[LE,RO]{}
\definecolor{tcc}{RGB}{217,217,217} % Table cell color

\renewcommand\tabularxcolumn[1]{m{#1}}
\setlength{\arrayrulewidth}{0.5pt}
\renewcommand{\arraystretch}{2}

% \renewcommand{\thesection}{\alph{section})}
% \renewcommand{\thesubsection}{\alph{section}.\arabic{subsection}}

\begin{document}

% ASEGÚRESE DE COLOCAR SUS DATOS EN LA PORTADA
\subfile{tex/portada}


\tableofcontents


%%% con \subfile se incluyen archivos externos
\subfile{tex/introduccion}
\subfile{tex/presentacion-estudiante}

\part*{Evidencias de aprendizaje}  % cuerpo del portafolio
\addcontentsline{toc}{part}{Evidencias de aprendizaje}
\markboth{Evidencias de aprendizaje}{Evidencias de aprendizaje}

%%%%%%%%%%%%%%%%%%%%%%%%%%%%%%%%%%%%%%%%%%%%%%%%%%%%%%%%%%%%%%%%%%%%%%
% Agregue su trabajo a partir de acá
%%%%%%%%%%%%%%%%%%%%%%%%%%%%%%%%%%%%%%%%%%%%%%%%%%%%%%%%%%%%%%%%%%%%%%
\subfile{tex/instrucciones}             % puede eliminar esta linea
\subfile{tex/ejemplo-derivada-numerica} % y esta

% Hasta acá

%%%%%%%%%%%%%%%%%%%%%%%%%%%%%%%%%%%%%%%%%%%%%%%%%%%%%%%%%%%%%%%%%%%%%%
% Conclusiones y referencias
\subfile{tex/conclusion-global}


%%%%%%%%%%%%%%%%%%%%%%%%%%%%%%%%%%%%%%%%%%%%%%%%%%%%%%%%%%%%%%%%%%%%%%
% Puede usar el capítulo de apéndice para agregar sus códigos completos si lo desea
% \appendix
% \subfile{tex/...}

% lista de referencias guardadas en referencias.bib
\bibliography{referencias}

\end{document}