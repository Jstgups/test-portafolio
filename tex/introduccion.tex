\documentclass[../portafolio.tex]{subfiles}

\begin{document}

% No edite este archivo

\chapter*{Introducción}
\addcontentsline{toc}{chapter}{Introducción}
\markboth{Introducción}{Introducción}

Este portafolio deberá incluir evidencias de las
actividades desarrolladas en la asignatura de Física Computacional II
(510240), dictada en el segundo semestre de 2024 en el departamento de
física, facultad de Ciencias Físicas y Matemáticas, de la Universidad
de Concepción.

\medskip

Esta asignatura se enfoca en la resolución de problemas en Física
usando métodos numéricos y el lenguaje de programación
\texttt{python}), contribuyendo a conocimientos generales de
construcción y aplicación de algoritmos computacionales, y
optimización de simulaciones numéricas. Así, se espera obtener los
siguientes resultados de aprendizaje al finalizar este portafolio:
\begin{enumerate}
\item Aplicar las herramientas computacionales en la resolución
  numérica de problemas en Física.
\item Generar programas computacionales apoyándose en algoritmos y
  conceptos de la física matemática y estadística.
\item Diferenciar, integrar y resolver ecuaciones diferenciales
  ordinarias, en forma numérica.
\end{enumerate}

\medskip

La evaluación se realizará a través de este portafolio, el cual debe
ser \textbf{entregado mediante un repositorio} de
\href{https://github.com}{GitHub} que será alojado en la organización
\href{https://github.com/fiscomp2-UdeC2024}{fiscomp2-UdeC2024}
especialmente habilitado para esta asignatura.

\begin{enumerate}
\item El repositorio será revisado periódicamente por el profesor o los ayudantes de
  la asignatura, para poder dar retroalimentación oportuna antes de la
  entrega oficial al final del curso. Las revisiones parciales se
  harán al finalizar cada mes, y la fecha estimada de entrega final es
  el \textbf{viernes 29 de noviembre de 2024}. Si el protafolio no
  tiene avances en cada revisión parcial, puede costar puntaje de la nota final.

\item El portafolio debe contener una sección de conclusiones donde
  el/la estudiante resumirá los principales logros de este portafolio
  y autoevaluará su desempeño a través de una reflexión final.
\end{enumerate}

\section*{Criterios de evaluación}
Como criterios de evaluación, se considerará:
\begin{enumerate}
\item Coherencia de las evidencias con las actividades realizadas en clases y la autoevaluación al final del documento.
\item Competencias comunicativas y redacción, especialmente uso
  correcto de ortografía, gramática, sintaxis, presentación de
  figuras, uso de elementos de \LaTeX\ y explicación de la parte
  relevante de scripts de \texttt{python}.
\item Presentación del documento: Claridad, limpieza y orden.
\end{enumerate}

Finalmente, este portafolio no solo es una herramienta de evaluación,
sino también una oportunidad para que el/la estudiante reflexione
sobre su proceso de aprendizaje, consolide sus conocimientos y
desarrolle habilidades clave en el ámbito de la computación aplicada a
la física. Se espera que este documento sirva como un registro
tangible de su progreso y como una muestra de las competencias
adquiridas a lo largo del curso, las cuales podrán ser de gran
utilidad en futuros desafíos académicos y profesionales.

\end{document}