\documentclass[../portafolio.tex]{subfiles}

% Solo agregue paquetes en el preámbulo de ../portafolio.tex

\begin{document}

\chapter*{Información personal y académica}
\addcontentsline{toc}{chapter}{Información personal y académica}
\markboth{Información personal y académica}{Información personal y académica}


%%%%%%%%%%%%%%%%%%%%%%%%%%%%%%%%%%%%%%%%%%%%%%%%%%%%%%%%%%%%%%%%%%%%%%
% Llene todos los campos, respetando tildes, mayúsculas y minúsculas.
\section*{Datos personales}

\begin{description}
\item[{Nombre completo}] Loki Laufeyson % nombres y apellidos completos.
\item[{Matrícula}] XXXXXX               % matrícula udec
\item[{Fecha de Nacimiento}] 17 de diciembre de 969 % día de mes de año
\item[{Nacionalidad}] Asgardiano
\item[{E-Mail institucional}] \href{mailto:loki@asgard.mcu}{loki@asgard.mcu}
\end{description}


%%%%%%%%%%%%%%%%%%%%%%%%%%%%%%%%%%%%%%%%%%%%%%%%%%%%%%%%%%%%%%%%%%%%%%
\section*{Breve biografía académica}
% Redacte una breve biografía (5 a 7 líneas) que incluya los
% siguientes aspectos:
% - Su nombre completo y el año en el que ingresó a la Universidad de
% Concepción.
% - Mencione su carrera actual y en qué año académico se encuentra.
% - Describa brevemente su trayectoria educativa previa a la universidad
% (por ejemplo, dónde cursó la educación media y cualquier logro académico
% relevante).
% - Mencione sus metas académicas y profesionales al finalizar el
% pregrado. ¿Qué le gustaría lograr al terminar la carrera? ¿En qué
% áreas le gustaría especializarse o trabajar?
% - Si lo considera pertinente, puede mencionar cualquier actividad
% extracurricular que haya contribuido a su formación (cursos,
% proyectos, trabajos, etc.).

Soy Loki Laufeyson, estudiante de segundo año de la carrera XXX. La educación media la realicé en el liceo/colegio XXX de la ciudad XXX....


%%%%%%%%%%%%%%%%%%%%%%%%%%%%%%%%%%%%%%%%%%%%%%%%%%%%%%%%%%%%%%%%%%%%%%
\section*{Visión general e interés sobre la asignatura}
% En esta sección, reflexione y describa:
% - ¿Cuál es su percepción inicial sobre la asignatura de Física
% Computacional II? ¿Cómo se relaciona con su formación académica y sus
% intereses?
% - ¿Qué habilidades o conocimientos espera desarrollar en esta
% asignatura, específicamente en el uso de herramientas computacionales
% aplicadas a la física?
% - ¿De qué manera cree que lo aprendido en esta asignatura contribuirá a
% su desempeño en otros cursos o en su carrera profesional a futuro?
% - Si tiene alguna expectativa específica o tema de interés particular
% dentro de la asignatura, menciónelo aquí.


%%%%%%%%%%%%%%%%%%%%%%%%%%%%%%%%%%%%%%%%%%%%%%%%%%%%%%%%%%%%%%%%%%%%%%
\section*{Resultados esperados de este portafolio}
% En esta sección, reflexione sobre los resultados que espera obtener al
% realizar este portafolio. Puede incluir lo siguiente:
% - ¿Qué habilidades y conocimientos espera haber consolidado al completar
% este portafolio?
% - ¿Cómo cree que el portafolio le ayudará a organizar, analizar y
% aplicar los conceptos aprendidos durante la asignatura?
% - ¿De qué manera considera que este portafolio puede servirle como
% referencia o herramienta para su futura formación académica o
% profesional?
% - Reflexione sobre cómo el proceso de autoevaluación y la inclusión de
% evidencias le permitirá comprender mejor su propio progreso.
\end{document}